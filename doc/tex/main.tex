\section{Wstęp}
Wykonanie realistycznej symulacji wiązało się z odpowiednim zamodelowaniem zjawisk fizycznych. W pierwszym rozdziale opisujemy w jaki sposób zasymulowaliśmy podstawowe prawa fizyki pozwalające na realistyczne zamodelowanie zjawisk zachodzących na stole bilardowym. W kolejnym rozdziale opisujemy jakich technik użyliśmy aby zaprezentować rezultat symulacji na ekranie. Rozdział opisuje w skrócie wykorzystane techniki renderingu przy użyciu biblioteki OpenGL oraz niskopoziomowe programowanie procesora grafiki przy użyciu techniki \textit{shaderów}.\\
Kolejne rozdziały opisują już działanie samego programu, jego obsługa raz testowanie.
