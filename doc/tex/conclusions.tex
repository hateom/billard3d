\section{Wnioski}

Podstawą w symulacjach zjawisk fizycznych jest odpowiednie zamodelowanie zjawisk oraz odpowiednie uproszczenie problemów, minimalizując koszty obliczeń. Ponieważ symulacja czasu rzeczywistego, aby działała poprawnie i realistycznie musi działać szybko. Obliczenia wykorzystujące zaawansowane operacje matematyczne takie jak funkcje trygonometryczne, pierwiastkowanie i tym podobne są bardzo czasochłonne dla procesora, więc pierwszą rzeczą jaką należy zrobić, to odpowiednie założenia, które pozwolą na uproszczenie zagadnienia, powodując zmniejszenie się poziomu trudności obliczeń. W naszym projekcie pierwszym założeniem upraszczającym obliczenia było ograniczenie wymiarów. Większość obliczeń jest wykonywana w dwóch wymiarach - w dwóch płaszczyznach (X oraz Y), pomimo, że renderowany obraz jest w trzech wymiarach. \\ \\

Bardzo istotnym elementem tego typu symulacji jest zastosowaniemetod numerycznych w obliczeniach. Zaokrąglenia, utrata precyzji mogą doprowadzić do nieprawidłowego działania symulacji, więc trzeba mieć to na uwadze. \\ \\

Istotnym elementem jest również poszukiwanie istniejących rozwiązań czyniących wiele rzeczy łatwiejszymi. Do takich należy zaliczyć rachunek kwaternionowy wykorzystany przy obliczeniach obrotów bil. Nie wykorzystując go udało y się uzyskać podobny rezultat, ale wiązało by się to z trudnymi oraz bardzo czasochłonnymi obliczeniami. \\ \\

Możliwości rozwoju programu są bardzo duże, i praktycznie w każdym miejscu symulacji można zwiększyć realizm modelując dokładniej zjawiska zachodzące w przyrodzie. Jednakże czas jakim dysponowaliśmy na wykonanie projektu pozwolił na podstawowe zasymulowanie zjawisk fizycznych, nie mniej jednak są one wystarczające do pokazania zjawisk, oraz do przestawienia jak w prosty sposób można zamodelować rozmaite zjawiska fizyczne.