\section{Obsługa i działnie programu}
\subsection{Obsługa programu}

Interakcja w programie zachodzi wyłącznie przy użyciu klawiatury. 
Podstawowe klawisze służące do sterowania przebiegiem symulacji to:
\begin{itemize}
 \item spacja - uderzenie bili - im dłużej przytrzymana spacja, tym siła uderzenia jest większa
 \item enter - zmiana kamery
 \item strzałki lewo prawo - zmiana celownika strzału
 \item escape - menu
\end{itemize}


Menu w każdej chwili dostępne jest po przyciśnięciu klawisza \textbf{ESC}.
Z poziomu menu mamy dostęp do opcji symulacji: włączenia bądź wyłączenia renderingu z użyciem shaderów, oraz
edycję podstawowych stałych, takich jak współczynnik tarcia oraz współczynnik sprężystości.

\subsection{Opcjonalne parametry}

Program w chwili obecnej interpretuje 2 parametry z linii poleceń:
\begin{itemize}
 \item --verbose który na standardowym wyjściu umieszcza informacje a temat działania symulacji
 \item --shaders=[1,0] określa, czy program ma wykorzystywać shadery (przydatne na komputerach z kartami, które słabo wspierają shadert) 
\end{itemize}

\subsection{Debugowanie}

Program posiada wiele systemów wspomagających debugowanie programu. Zostały zaimplementowane odpowiednie makra tworzące
pewien specyficzny interfejs tworzenia kodu, który generuje informacje przydatne przy nieoczekiwanym przerwaniu działania programu. Domyślenie projekt obsługuje wyjątki, i na ich bazie stworzony został framework do śledzenia rzucanych wyjątków. W momencie wystąpienia sytuacji krytycznej i rzucenia wątku cały stos jest zapisywany w pliku tekstowym, z którego można prześledzić dokładnie w którym miejscu wydarzył się wyjątek.